%%%%%%%%%%%%%%%%%%%%%%%%
% Sample use of the infthesis class to prepare an MSc thesis.
% This can be used as a template to produce your own thesis.
% Date: June 2019
%
%
% The first line specifies style options for taught MSc.
% You should add a final option specifying your degree.
% *Do not* change or add any other options.
%
% So, pick one of the following:
% \documentclass[msc,deptreport,adi]{infthesis}     % Adv Design Inf
% \documentclass[msc,deptreport,ai]{infthesis}      % AI
% \documentclass[msc,deptreport,cogsci]{infthesis}  % Cognitive Sci
% \documentclass[msc,deptreport,cs]{infthesis}      % Computer Sci
% \documentclass[msc,deptreport,cyber]{infthesis}   % Cyber Sec
% \documentclass[msc,deptreport,datasci]{infthesis} % Data Sci
% \documentclass[msc,deptreport,di]{infthesis}      % Design Inf
% \documentclass[msc,deptreport,dsti]{infthesis}    % Data Sci TI
% \documentclass[msc,deptreport,inf]{infthesis}     % Informatics
%%%%%%%%%%%%%%%%%%%%%%%%

\documentclass[msc,deptreport,dsti]{infthesis} % Do not change except to add your degree (see above).
\usepackage{subfiles}
\usepackage{graphicx}
\usepackage{float}
\usepackage{subfigure}
\usepackage[toc,title,page]{appendix}
\usepackage[ruled,linesnumbered]{algorithm2e}

\begin{document}
\begin{preliminary}

\title{An Attempt to Reduce the Number of Training Samples for Convolutional Neural Networks}

\author{Yuwen Heng}

\abstract{
Training deep neural networks can be resources-consuming. The budget required is increasing with the size of the dataset. During the past ten years, many achievements are dedicated to accelerating the convergence speed with heuristic or theoretical training procedures.  However, we still need the whole dataset to train the network and paying for a large dataset may not pay back well if we can use a smaller subset to achieve an acceptable performance. In order to reduce the number of training samples needed, we first adapted and evaluated three methods, Patterns by Ordered Projections (POP), Enhanced Global Density-based Instance Selection (EGDIS), and Curriculum Learning (CL), to reduce the size of two image datasets, CIFAR10 and CIFAR100, for the classification task. Based on the analysis, we present our main contributions: improved CL and evaluated its two variations, the Weighted Curriculum Learning (WCL) and the Boundary based Weighted Curriculum Learning (BWCL). The WCL outperforms POP and EGDIS in terms of both classification accuracy and time complexity. Also, WCL and BWCL achieve comparable performance compared with CL while keeping a portion of hard examples. Besides, we proposed a trade-off framework for WCL to select a subset of samples according to the acceptable relative accuracy and the original datasets. 
}

\maketitle

\section*{Acknowledgements}
First and foremost, I would like to express my deepest gratitude to my supervisor, Dr Yang Cao, for offering me the opportunity to work with him on such an attracting and challenging project. His encouragement and valuable guidance helped me tackle the obstacles in my research path. 

Furthermore, special thanks go to Professor Bob Fisher, Dr Pavlos Andreadis at the University of Edinburgh and Dr Jiacheng Ni at IBM for sharing me their knowledge about computer vision and deep learning. The programming skills and coursework experience that I learnt from them helped me to organise the experiments well.

Finally, I would like to send my love to my fiancee Danni Li for her accompany during the past three years. I wouldn't have the chance to study full-time without her full support. 
\tableofcontents
\end{preliminary}


\chapter{Introduction}
\subfile{sections/cp1-introduction.tex}

\chapter{Background Research}
\label{bg}
\subfile{sections/cp2-background.tex}

\chapter{Adapted Data Reduction Methods}
\subfile{sections/cp3-methodology.tex}

\chapter{Data Reduction Evaluations}
\subfile{sections/cp4-revaluation.tex}

\chapter{Trade-off Framework}
\subfile{sections/cp5-tfframework.tex}

\chapter{Conclusion and Future Work}
\subfile{sections/conclusion}


\bibliographystyle{plain}
\bibliography{library.bib}



%% You can include appendices like this:
 \appendix
 
 \chapter{Training History}
 \subfile{sections/appendix}

% 
% Markers do not have to consider appendices. Make sure that your contributions
% are made clear in the main body of the dissertation (within the page limit).

\end{document}
