%%%%%%%%%%%%%%%%%%%%%%%%
% Sample use of the infthesis class to prepare an MSc thesis.
% This can be used as a template to produce your own thesis.
% Date: June 2019
%
%
% The first line specifies style options for taught MSc.
% You should add a final option specifying your degree.
% *Do not* change or add any other options.
%
% So, pick one of the following:
% \documentclass[msc,deptreport,adi]{infthesis}     % Adv Design Inf
% \documentclass[msc,deptreport,ai]{infthesis}      % AI
% \documentclass[msc,deptreport,cogsci]{infthesis}  % Cognitive Sci
% \documentclass[msc,deptreport,cs]{infthesis}      % Computer Sci
% \documentclass[msc,deptreport,cyber]{infthesis}   % Cyber Sec
% \documentclass[msc,deptreport,datasci]{infthesis} % Data Sci
% \documentclass[msc,deptreport,di]{infthesis}      % Design Inf
% \documentclass[msc,deptreport,dsti]{infthesis}    % Data Sci TI
% \documentclass[msc,deptreport,inf]{infthesis}     % Informatics
%%%%%%%%%%%%%%%%%%%%%%%%

\documentclass[msc,deptreport,dsti]{infthesis} % Do not change except to add your degree (see above).

\begin{document}
\begin{preliminary}

\title{Evaluating data reduction techniques for supervised training}

\author{Yuwen Heng}

\abstract{
Training deep neural networks can be resources-consuming. The budget required are increasing with the size of the dataset. During the past few decades, many research is dedicated to develop training procedures to accelerate the convergence speed of deep learning. However, we still need the whole dataset to train the network and paying for a large dataset may not pay back well if we can use a smaller subset to achieve an acceptable performance. To solve this issue, we first adapted and evaluated three methods, Patterns by Ordered Projections (POP), Enhanced Global Density-based Instance Selection (EGDIS), and Curriculum Learning (CL), to reduce the size of two image datasets, CIFAR10 and CIFAR100, for the classification task. Based on the analysis, we present our two contributions: the Weighted Curriculum Learning (WCL) and a trade-off framework. The WCL outperforms POP and EGDIS in terms of both classification accuracy and time complexity. It achieves comparable performance compared with CL while keeping a portion of hard examples. The trade-off framework selects a subset of samples according to the acceptable relative accuracy and the dataset. In addition, the framework is also extended to predict the amount of samples needed to achieve a particular accuracy with a given subset.
}

\maketitle

\section*{Acknowledgements}
Any acknowledgements go here. 

\tableofcontents
\end{preliminary}


\chapter{Introduction}
Besides, paying for a large dataset may not pay back well if we can use a much smaller one to achieve an acceptable performance.
The preliminary material of your report should contain:
\begin{itemize}
\item
The title page.
\item
An abstract page.
\item
Optionally an acknowledgements page.
\item
The table of contents.
\end{itemize}

As in this example \texttt{skeleton.tex}, the above material should be
included between:
\begin{verbatim}
\begin{preliminary}
    ...
\end{preliminary}
\end{verbatim}
This style file uses roman numeral page numbers for the preliminary material.

The main content of the dissertation, starting with the first chapter,
starts with page~1. \emph{\textbf{The main content must not go beyond page~40.}}

The report then contains a bibliography and any appendices, which may go beyond
page~40. The appendices are only for any supporting material that's important to
go on record. However, you cannot assume markers of dissertations will read them.

You may not change the dissertation format (e.g., reduce the font
size, change the margins, or reduce the line spacing from the default
1.5 spacing). Over length or incorrectly-formatted dissertations will
not be accepted and you would have to modify your dissertation and
resubmit.  You cannot assume we will check your submission before the
final deadline and if it requires resubmission after the deadline to
conform to the page and style requirements you will be subject to the
usual late penalties based on your final submission time.

\section{Using Sections}

Divide your chapters into sub-parts as appropriate.

\section{Citations}

Citations (such as \cite{P1} or \cite{P2}) can be generated using
\texttt{BibTeX}. For more advanced usage, the \texttt{natbib} package is
recommended. You could also consider the newer \texttt{biblatex} system.

These examples use a numerical citation style. You may also use
(Author, Date) format if you prefer.

\chapter{Your next chapter}

A dissertation usually contains several chapters.

\chapter{Conclusions}

\chapter{New file}
I want to know what is it.

\section{Final Reminder}

The body of your dissertation, before the references and any appendices,
\emph{must} finish by page~40. The introduction, after preliminary material,
should have started on page~1.

You may not change the dissertation format (e.g., reduce the font
size, change the margins, or reduce the line spacing from the default
1.5 spacing). Over length or incorrectly-formatted dissertations will
not be accepted and you would have to modify your dissertation and
resubmit.  You cannot assume we will check your submission before the
final deadline and if it requires resubmission after the deadline to
conform to the page and style requirements you will be subject to the
usual late penalties based on your final submission time.

\bibliographystyle{plain}
\bibliography{mybibfile}

%% You can include appendices like this:
% \appendix
% 
% \chapter{First appendix}
% 
% \section{First section}
% 
% Markers do not have to consider appendices. Make sure that your contributions
% are made clear in the main body of the dissertation (within the page limit).

\end{document}
